\documentclass[a4paper, 10pt, french]{article}
% Préambule; packages qui peuvent être utiles
   \RequirePackage[T1]{fontenc}        % Ce package pourrit les pdf...
   \RequirePackage{babel,indentfirst}  % Pour les césures correctes,
                                       % et pour indenter au début de chaque paragraphe
   \RequirePackage[utf8]{inputenc}   % Pour pouvoir utiliser directement les accents
                                     % et autres caractères français
   \RequirePackage{lmodern,tgpagella} % Police de caractères
   \textwidth 17cm \textheight 25cm \oddsidemargin -0.24cm % Définition taille de la page
   \evensidemargin -1.24cm \topskip 0cm \headheight -1.5cm % Définition des marges
   \RequirePackage{latexsym}                  % Symboles
   \RequirePackage{amsmath}                   % Symboles mathématiques
   \RequirePackage{tikz}   % Pour faire des schémas
   \RequirePackage{graphicx} % Pour inclure des images
   \RequirePackage{listings} % pour mettre des listings
% Fin Préambule; package qui peuvent être utiles

\title{Rapport de TP 4MMAOD : Génération de patch optimal}
\author{
    LEROUX Benjamin (ISI1)
\\  THALAMY Pierre (ISI2)
}

\begin{document}

\maketitle

%%%%%%%%%%%%%%%%%%%%%%%%%%%%%%%%%%%%%%%%%%%%%%
\paragraph{\em Préambule}
    Nous avons choisi d'écrire notre programme de calcul de patch optimal en C++ pour des raisons de performance. En effet, après une première version du code en Python, nous n'étions pas satisfait du temps d'exécution de notre programme. Le C++ nous offre un compromis intéressant entre performance brut et facilité d'implémentation au moyen des classes de base.

%%%%%%%%%%%%%%%%%%%%%%%%%%%%%%%%%%%%%%%%%%%%%%
\section{Principe de notre  programme (1 point)}

Notre programme remplit deux matrices de tailles égales aux nombres de lignes + 1 présendans les fichiers source et cible. Chaque case (i, j) de la première matrice représente le coût minimal du patch transformant le fichier source en fichier cible jusqu'aux lignes i et j respectivement. La deuxième matrice conserve en mémoire le dernier choix optimal réalisé pour arriver jusqu'à ces lignes.

Ensuite on parcourt la matrice des choix en partant de la case de plus haut indices, puis on détermine les opérations effectuées et on décrémente en fonction les indices (i,j) afin d'arriver au point (0, 0).

Nous avons utilisé une méthode itérative dans toutes nos boucles.

%%%%%%%%%%%%%%%%%%%%%%%%%%%%%%%%%%%%%%%%%%%%%%
\section{Analyse du coût théorique (3 points)}
Nous avons remarqué que l'essentiel du temps d'exécution de notre algorithme correspond au temps passé au remplissage des matrices. En effet en dehors du remplissage, on ne passe que peu de temps à lire les fichiers, initialiser les matrices ou même écrire le patch. De plus on effectue 12 opérations élémentaires dans les boucles imbriqués de remplissage.\\
Lecture : $O(n_1+n_2)$\\
Initialisation : $O(n_1+n_2)$\\
Écriture du patch : $O(\sqrt{n_1+n_2})$ \\
Remplissage des matrices : $O(12.n_1.n_2)$\\
Coût total théorique : $O(2(n_1+n_2) + \sqrt{n_1+n_2} + 12.n_1.n_2)$

  \subsection{Nombre  d'opérations en pire cas\,: $3.n_1 + 3.n_2 + 14.n_1.n_2$}
    \paragraph{Justification\,: }
    Le programme itératif contient les boucles \textit{while} : $k_1=1..n_1$, $k_2= 1..n_2$ correspondant à la lecture des lignes fichiers, puis on initialise nos matrices grâce à des boucles \textit{for} : $k_3=0..n_1$, $k_4= 0..n_2$, et nous remplissons des matrices de tailles ($n_1, n_2$) grâce à des boucles imbriqués \textit{for} : $k_5=1..n_1$, $k_6= 1..n_2$. Ensuite nous parcourons une matrice de taille ($n_1, n_2$), ce qui prend au pire des cas $n_1 + n_2$ opérations. Enfin nous devons afficher le patch ce qui peut prendre jusqu'à $2n_1.n_2$ opérations.
    \begin{align}
      T(n_1, n_2) &= \sum_{k_1=1}^{n_1} O(1) + \sum_{k_2=1}^{n_2} O(1) + \sum_{k_3=0}^{n_1} O(1) + \sum_{k_4=0}^{n_2} O(1)\\
      & \;+ n_1 + n_2 + 2.n_1.n_2 \\
      & \;+ \sum_{k_5=1}^{n_2} \sum_{k_6=1}^{n_1} \underbrace{O(1)}_{\text{Init des coef}} + \underbrace{2.O(1)}_{\text{Calcul Min D}} + \underbrace{4.O(1)}_{\text{Calcul min}} + \underbrace{4.O(1)}_{\text{Sauv de l'opération}} + \underbrace{O(1)}_{\text{Sauv nb de lignes D}}\\
      &= 3.n_1 + 3.n_2 + 14.n_1.n_2
    \end{align}

  \subsection{Place mémoire requise\,: $2.n_1.n_2 + O(\sqrt{n_1+n_2}) + O(n_1) + O(n_2)$}
    \paragraph{Justification\,: }
    On utilise deux matrices pour stocker les coûts précédemment calculés, et les choix effectués. De plus on charge les deux fichiers afin de ne pas avoir à effectuer de lecture durant les boucles.
    On utilise aussi un tableau de string pour stocker le patch calculé avant de l'afficher (de taille $O(\sqrt{n_1+n_2})$).

  \subsection{Nombre de défauts de cache sur le modèle CO\,: $2.n_1. n_2 + n_1$}
    \paragraph{Justification\,: }
    Nous remplissons les matrices colonne par colonne alors qu'elles sont stockées lignes par lignes ce qui nous contraint à effectuer deux défauts de cache à chaque tour de nos boucles imbriqués. On considère que L est inférieur à $n_2$.

    De plus on parcourt la matrice des choix lors du calcul du patch, ce qui nous fait effectuer $n_1$ changement de ligne, et donc autant de changement ligne. On considère ici qu'on ne fait pas de saut de distance supérieur à L dans une ligne.

%%%%%%%%%%%%%%%%%%%%%%%%%%%%%%%%%%%%%%%%%%%%%%
\section{Compte rendu d'expérimentation (2 points)}
  \subsection{Conditions expérimentales}

    \subsubsection{Description synthétique de la machine\,:}
    La machine a été mise en mode mono-utilisateur, sans interface graphique, et notre test était donc le seul processus actif. Elle a les specs suivantes: \\
    \textbf{OS:} Antergos (Arch) Linux 64-bit\\
    \textbf{Mémoire RAM:} 3,8 GB\\
    \textbf{Processeur:} CPU Intel Core M 5Y10 @0.800GHz (Turbo -> 2 GHz) x 4\\
    \textbf{Compilateur:} g++ version 5.2\\

    \subsubsection{Méthode utilisée pour les mesures de temps\,: }
    Le benchmark a été effectué à l'aide du script \textit{benchmark.sh} situé dans le répertoire du projet, il ne prend pas d'arguments et se contente d'exécuter 5 fois consécutives \textbf{computePatchOpt} sur chacun des dossiers de benchmark du répertoire \textit{benchmark/} avec l'utilitaire UNIX \textbf{time}. Il affiche sur la sortie standard le temps en seconde de type \textit{real} (Temps d'exécution du programme avec les appels systèmes nécessaires au programme).

  \subsection{Mesures expérimentales}
    Les temps ci-dessous sont en secondes et comprennent l'exécution du programme, comprenant les appels système.

    \begin{figure}[h]
      \begin{center}
        \begin{tabular}{|l||r||r|r|r||}
          \hline
          \hline
            & coût         & temps     & temps   & temps \\
            & du patch     & min       & max     & moyen \\
          \hline
          \hline
            benchmark1 &   2540   & 0,035  &  0,035  &  0,035 \\
          \hline
            benchmark2 &   3120   & 0,204  &  0,205  &  0,2044 \\
          \hline
            benchmark3 &   809   &  0,291  &  0,293  &  0,2922 \\
          \hline
            benchmark4 &   1708  & 0,759  &  0,762  &  0,7604 \\
          \hline
            benchmark5 &   7553   & 1,304  &  1,324  &  1,3084 \\
          \hline
            benchmark6 &   37027  & 7,016  &  7,035  &  7,0276  \\
          \hline
          \hline
        \end{tabular}
        \caption{Mesures des temps minimum, maximum et moyen de 5 exécutions pour les 6 benchmarks.}
        \label{table-temps}
      \end{center}
    \end{figure}

\subsection{Analyse des résultats expérimentaux}
On remarque que $n_1$ et $n_2$ sont très proche. On peut alors prendre $N = \max{n_1, n_2}$ et grâce aux calculs théoriques estimer que le temps d'exécution est fonction de N de cette façon : $16.N^2 +7.N$  (Ajout du max des opérations effectuées et des défauts de cache)

Cependant lorsque le nombre de ligne max des fichiers double on devrait multiplier le temps d'exécution par un facteur de $\frac{78}{23} \simeq 3,4$. Or on remarque plutôt que le temps d'exécution est multiplié par 6. On en conclut que les résultats théoriques et expérimentaux ne coïncide pas, soit parce que nos calculs de coût sont trop approximatif, soit que d'autres variables entrent dans le calcul du coup d'exécution.

\end{document}
%% Fin mise au format
